Foi construído um museu de arte moderna na cidade de Timóteo. Com o intuito de garantir a segurança das obras
de arte ali presentes, câmeras com visão $360 ^{\circ}$ foram instaladas pelo museu.
Devido ao corte de custos, apenas uma câmera foi instalada por sala.

A empresa que realizou a instalação não certificou se todas as paredes eram completamente visíveis pela câmera e agora ela precisa de sua ajuda.

Dado a planta baixa de uma sala do museu e a posição da câmera, você deve indicar se todas as paredes são visíveis por ela ou não.

Uma parede é considerada vigiada pela câmera se todos os pontos sobre ela podem ser vistos pela câmera, ou seja, não são bloqueados por outra parede.
Uma parede colinear com a câmera também não é considerada como vigiada.


\section*{Entrada}

A entrada é composta pela descrição de uma sala do museu e da posição da câmera, sendo que a sala é
representada por um polígono simples.
A primeira linha contêm um inteiro $N$, que representa a quantidade de paredes da sala.
Em seguida, temos $N$ linhas. A $i$-ésima linha contêm $2$ inteiros $X_i$, $Y_i$ que são as coordenadas do $i$-ésimo ponto.
Uma parede é formada pela ligação do ponto $i$ com o ponto $i+1$. O $N$-ésimo ponto da entrada com o primeiro ponto formam a última parede.
Por fim, temos mais uma linha com dois inteiros $C_x, C_y$, que representam as coordenadas da câmera.

\section*{Saída}

A saída deve conter uma linha com o caracter \texttt{S} caso todas as paredes sejam completamente vigiadas pela câmera e \texttt{N} caso contrário.

\section*{Restrições}

\begin{itemize}
    \item $3 \leq N \leq 10^5 $
    \item $-10^5 \leq X_i, Y_i, C_x, C_y \leq 10^5 $
\end{itemize}

É garantido que as coordenadas geram um polígono simples e que as coordenadas da câmera se encontram dentro desse polígono.

\section*{Exemplos}

\exemplo


\begin{figure}[!h]
\centering
\begin{tikzpicture}

    % Draw the Axis
    \draw[thick,->] (0,0) -- (7,0) node[anchor=north west] {x};
    \draw[thick,->] (0,0) -- (0,7) node[anchor=south east] {y};

    \foreach \x in {0,1,2,3,4,5,6}
        \draw (\x cm,1pt) -- (\x cm,-1pt) node[anchor=north] {$\x$};
    \foreach \y in {0,1,2,3,4,5,6}
        \draw (1pt,\y cm) -- (-1pt,\y cm) node[anchor=east] {$\y$};
    

    \coordinate (CAM1) at (4,1);
    \draw (CAM1) node[anchor=south west] {X} node {$\bullet$};

    \coordinate (CAM2) at (5,3);
    \draw (CAM2) node[anchor=south west] {Y} node {$\bullet$};
    
    \coordinate (CAM3) at (5,5);
    \draw (CAM3) node[anchor=south west] {Z} node {$\bullet$};

    \draw (0,0) -- (6,0) -- (6,6) -- (3,6) -- (3,3) -- (0,3) -- (0,0);
    \draw [draw=black!90,very thick](3,3) -- (0,3);

    \draw (0,0) node[anchor=south west] {a} node {$\bullet$};
    \draw (6,0) node[anchor=south west] {b} node {$\bullet$};
    \draw (6,6) node[anchor=south west] {c} node {$\bullet$};
    \draw (3,6) node[anchor=south west] {d} node {$\bullet$};
    \draw (3,3) node[anchor=south west] {e} node {$\bullet$};
    \draw (0,3) node[anchor=south west] {f} node {$\bullet$};


\end{tikzpicture}
\caption{Os pontos $X$, $Y$ e $Z$ representam a posição da câmera no primeiro, segundo e terceiro exemplo, respectivamente.
No segundo exemplo, a câmera $Y$ não consegue monitorar completamente a parede $(e,f)$ e no terceiro exemplo a
câmera $Z$ não monitora $(e,f)$ e $(f, a)$ }
\end{figure}
