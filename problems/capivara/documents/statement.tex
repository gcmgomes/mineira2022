Sim, isso está acontecendo de novo.

Pela terceira maratona seguida teremos um problema envolvendo a dinastia fundada por Bacon -- o Grade.
Desta vez, o protagonista de nossa história é o neto mais novo de Bacon -- o Bravo -- filho de Bacon -- a Ambiciosa -- e o rei mais improvável já registrado como monarca de todas as capivaras.

Esta passagem das \textit{Crônicas das Capivaras} narra os primeiros momentos do reinado de Bacon -- o Grafo -- que, ao invés de se tornar um monarca absolutista como seus antecessores, repartiu o poder entre vários ministros para poder se dedicar a sua verdadeira paixão: a combinatória. Ele voltou suas atenções para a abertura de um antigo baú do tesouro adquirido por sua mãe após uma dura campanha de inverno contra insurgentes e que tem frustrado os mais renomados pensadores do reino.

A fechadura do baú é composta por um tabuleiro quadriculado de dimensões $N$ por $M$ em que cada umas das $N*M$ posições tem uma joia encrustada que pode mudar para uma de $3$ possíveis cores.

O mecanismo de segurança funciona da seguinte forma: se existe um conjunto de joias da mesma cor de tamanho pelo menos $K$ no qual podemos traçar um caminho entre quaisquer duas que passe apenas pelo conjunto, as joias desaparecem; a trava é liberada somente se todas as joias desaparecerem ao mesmo tempo.

Exausto após tantas tentativas mal sucedidas, nosso herói caminhou para uma necessária noite de sono; ele não sabia nem se era possível abrir o tesouro e as dúvidas o deixaram cada vez mais desesperançoso.
Mas por quê necessária?
Bom, está escrito nas \textit{Crônicas} que Bacon -- o Grafo -- recebeu, durante o sonho, a resposta de três entidades místicas que se auto intitulavam "maratonistas".

Claramente, vocês são essas entidades. Salvem nosso monarca favorito lhe dizendo se o tesouro pode ou não ser aberto e, caso possa, qual o menor número de joias que devem ter a cor trocadas para tal.


\section*{Entrada}

A primeira linha da entrada contem os inteiros $N, M$ e $K$, separados por espaço, que representam, respectivamente, o número de linhas e colunas do tabuleiro e o tamanho mínimo para o conjunto de joias monocromáticas desaparecer.
Em seguida, existem $N$ linhas, cada uma com $M$ inteiros, separados por espaço, entre $1$ e $3$, representando as cores das joias na respectiva linha.
É garantido que o tamanho dos conjuntos de mesma cor na entrada não excedem $K$.

\section*{Saída}

A saída contem um único inteiro $M$, o menor número de recolorações que devem ser feitas para abrir o baú.
Caso isso seja impossível, imprima $-1$.

\section*{Restrições}

\begin{itemize}
\item $1 \leq N, M \leq 5 * 10^7$
\item $1 \leq N*M \leq 5 * 10^7$
\item $2 \leq K \leq 5*10^7$
\end{itemize}


\section*{Exemplos}

\exemplo
