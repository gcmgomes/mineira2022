Sim, isso está acontecendo de novo.

Pela terceira maratona seguida teremos um problema envolvendo a dinastia fundada por Bacon -- o Grande.
Desta vez, o protagonista de nossa história é o neto mais novo de Bacon -- o Bravo -- filho de Bacon -- a Ambiciosa -- e o rei mais improvável já registrado como monarca de todas as capivaras.

Esta passagem das \textit{Crônicas das Capivaras} narra os primeiros momentos do reinado de Bacon -- o Grafo -- que, ao invés de se tornar um monarca absolutista como seus antecessores, repartiu o poder entre vários ministros para poder se dedicar a sua verdadeira paixão: a combinatória.
Ele voltou suas atenções para a abertura de um antigo baú adquirido por sua mãe após uma dura campanha de inverno contra insurgentes e que tem frustrado os mais renomados pensadores do reino.

A fechadura do baú é composta por um tabuleiro quadriculado de dimensões $N$ por $M$ em que cada umas das $N \times M$ posições tem uma joia encrustada que pode mudar para uma de $3$ possíveis cores.
Uma joia na posição $(a,b)$ do tabuleiro é adjacente a todas as joias $(x,y)$ que satisfazem $|a - x| + |b - y| = 1$.
As joias tem propriedades mágicas, de tal forma que duas joias em posições adjacentes no tabuleiro entram em ressonância se e somente se elas tem a mesma cor.
Esse efeito é transitivo, ou seja, se $(a,b)$ está em ressonância com $(c,d)$ e $(c,d)$ está em ressonância com $(x,y)$, então $(a,b)$ também está em ressonância com $(x,y)$.
As joias, porém, não são muito resistentes e se uma joia está em ressonância com pelo menos outras $K$ joias, ela se desintegra instantaneamente.
O tabuleiro do baú também é especial, e permite que seu usuário troque a cor de uma única joia por vez.

O mecanismo de segurança é simples: o tesouro é destrancado se e somente se todas as joias no tabuleiro se desintegram no mesmo instante de tempo.

Exausto após tantas tentativas mal sucedidas, nosso herói caminhou para uma necessária noite de sono; ele não sabia nem se era possível abrir o tesouro e as dúvidas o deixaram cada vez mais desesperançoso.
Mas por quê necessária?
Bom, está escrito nas \textit{Crônicas} que Bacon -- o Grafo -- recebeu, durante o sonho, a resposta de três entidades místicas que se auto intitulavam "maratonistas".

Claramente, vocês são essas entidades. Salvem nosso monarca favorito lhe dizendo se o tesouro pode ou não ser aberto e, caso possa, qual o menor número de joias que devem ter a cor trocadas para tal.


\section*{Entrada}

A primeira linha da entrada contém os inteiros $N, M$ e $K$, separados por espaço, que representam, respectivamente, o número de linhas e colunas do tabuleiro e o limite de resistência das joias.
Em seguida, existem $N$ linhas, cada uma com $M$ inteiros, separados por espaço, representando as cores das joias na respectiva linha.
É garantido que nenhuma joia está em ressonância com $K$ outras joias.

\section*{Saída}

A saída contem um único inteiro $M$, o menor número de recolorações que devem ser feitas para abrir o baú.
Caso isso seja impossível, imprima $-1$.

\section*{Restrições}

\begin{itemize}
\item $1 \leq N, M \leq 5 \times 10^6$.
\item $1 \leq N \times M \leq 5 \times 10^6$.
\item $1 \leq K \leq 5 \times 10^6$.
\end{itemize}


\section*{Exemplos}

\exemplo
