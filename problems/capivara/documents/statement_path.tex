Sim, isso está acontecendo de novo.
Pela terceira maratona seguida teremos um problema envolvendo a dinastia fundada por Bacon -- o Grade.
Desta vez, o protagonista de nossa história é o neto mais novo de Bacon -- o Bravo -- filho de Bacon -- a Ambiciosa -- e o rei mais improvável já registrado como monarca de todas as capivaras.
Esta passagem das \textit{Crônicas das Capivaras} narra os primeiros momentos do reinado de Bacon -- o Grafo -- que, ao invés de se tornar um monarca absolutista como seus antecessores, repartiu o poder entre vários ministros para poder se dedicar a sua verdadeira paixâo: a combinatória.
Ele voltou suas atenções para a abertura de um antigo tesouro adquirido por sua mãe após uma dura campanha de inverno contra insurgentes e que tem frustrado os mais renomados pensadores do reino.
A fechadura do baú é composta por uma sequência de $N$, cada uma inicialmente possui uma de $3$ cores.
O mecanismo de segurança funciona da seguinte forma: se existe uma subsequência contígua de joias da mesma cor de tamanho pelo menos $K$, as joias desaparecem; a trava é liberada se somente se todas as joias desaparecerem.
Capiváras são quadrúpedes, e por isso o monarca só pode trocar a cor de uma joia por vez.
Exausto após tantas tentativas mal sucedidas, nosso herói caminhou exausto para uma necessária noite de sono; ele não sabia nem se era possível abrir o tesouro, e as dúvidas o ddeixaram cada vez mais desesperançoso.
Mas por quê necessária?
Bom, está escrito nas \textit{Crônicas} que Bacon -- o Grafo -- recebeu a resposta de três entidades místicas que se autointitulavam "maratonistas" durante um sonho.
Claramente, vocês são essas entidades.
Salvem nosso monarca favorito da insanidade lhe dizendo se o tesouro pode ou não ser aberto e, caso possa, qual o menor número de trocas de cores de joias que devem feitas para tal.


\section*{Entrada}

A primeira linha da entrada contem os inteiros $N$ e $K$, separados por espaço, que representam, respectivamente, o número de joias na fechadura  e o tamanho mínimo para um conjunto de joias monocromáticas desaparecer.
Em seguida, existe uma linha com $N$ inteiros, separados por espaço, entre $1$ e $3$, representando as cores das joias.
É garantido que o tamanho dos conjuntos de mesma cor na entrada não excedem $K$.

\section*{Saída}

A saída contem um único inteiro $M$, o menor número de recolorações que devem ser feitas para abrir o baú.
Caso isso seja impossível, imprima $-1$.

\section*{Restrições}

\begin{itemize}
\item $1 \leq N \leq 10^5$
\item $1 \leq K \leq 30$
\end{itemize}


\section*{Exemplos}

\exemplo
