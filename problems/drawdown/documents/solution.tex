Com algumas manipuções algébricas e uma análise de casos, é possível usar uma SegTree para resolver o problema. 

Considere um intervalo [L, R]. Uma vez que X[i] >= 1, temos que:

max{(x[i] - x[j]) / x[i] : L <= i <= j <= R} =  max{1 - x[j]/x[i] : L <= i <= j <= R} = 1 - min{x[j]/x[i] : L <= i <= j <= R}.  Então, vamos considerar o problema de calcular o T(L, R) = min{x[j]/x[i] : L <= i <= j <= R}.

Para poder usar uma SegTree, precisamos ser capazes de juntar resultados parciais de dois intervalos menores para formar os resultados parciais de um intervalo maior. Considere que o intervalo [L, R] foi dividido em dois intervalos: [L, M] e [M + 1, R]. Assuma que, além de T(L, M) e T(M + 1, R), sabemos o maior e o menor valor em cada um desses intervalos.

Seja i* e j* índices tais que T(L, R) = x[j*] / x[i*]. Não conhecemos i* e j*, mas sabemos que os casos possíveis são:

1) Se L <= i* <= j* <= M, T(L, R) = T(L, M).
2) Se M + 1 <= i* <= j* <= R, T(L, R) = T(M + 1, R).
3) Se L <= i* <= M < j* <= R, T(L, R) = min{X[M + 1], .., X[R]} / max{X[L], ..., X[M]}.

Então, T(L, R) = min{T(L, M), T(M + 1, R), min{X[M + 1], .., X[R]} / max{X[L], ..., X[M]}}.
