Pedro é um artista pop que anda fazendo muito sucesso no Brasil. Suas músicas estao em todas as rádios, e os cachês de seus shows estão cada vez mais altos! Ele está realmente no topo de sua carreira.

Pedro está planejando uma grande turnê, que irá rodar o mundo todo. Porém, ele está tendo dificuldade em definir os dias dos seus shows. A cada dia o valor do cache de Pedro varia (depende do dia da semana, mes, previsão do tempo do dia, etc...). Ele quer organizar os shows de modo que seu cache seja o mais alto possível, porém ele não quer fazer shows em dias consecutivos, pois isso o deixaria muito cansado e ele não conseguiria dar o seu melhor.  Você pode ajudá-lo com isso?

Dada a quantidade de dias que durará a turnê e o valor do cache de cada dia, ajude Pedro a encontrar o maior cache possível que ele pode receber somados todos os shows, lembrando que Pedro não pode realizar shows em dias consecutivos.

\section*{Entrada}

A primeira linha da entrada contem um inteiro $N$, a quantidade de dias que durara a turnê. As próximas $N$ linhas seguintes contém cada uma um inteiro $K[i]$, representando o valor do cache de Pedro no dia $i$.

\section*{Saída}

A saída deve conter um único inteiro X, o valor máximo do cache que Pedro pode receber na turnê.

\section*{Restrições}

\begin{itemize}
\item $1 \leq N \leq 10^5$
\item $1 \leq K[i] \leq 10^9$
\end{itemize}


\section*{Exemplos}

\exemplo
