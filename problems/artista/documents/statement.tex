Pedro é um artista pop que anda fazendo muito sucesso no Brasil. Suas músicas estão em todas as rádios e os cachês de seus shows estão cada vez mais altos! Ele está realmente no topo de sua carreira.

Pedro está planejando uma grande turnê que irá rodar o mundo todo, porém está com dificuldade em definir os dias dos seus shows. O valor do cachê do Pedro varia diariamente (depende do dia da semana, mês, previsão do tempo do dia, etc...). Ele quer organizar os shows de modo que seu cachê seja o mais alto possível, porém ele não quer fazer shows em dias consecutivos, pois isso o deixaria muito cansado e ele não conseguiria dar o seu melhor.  Você pode ajudá-lo com isso?

Dada a quantidade de dias que durará a turnê e o valor do cachê de cada dia, ajude Pedro a encontrar o maior valor possível que ele pode receber somados todos os shows, lembrando que Pedro não pode realizar shows em dias consecutivos.

\section*{Entrada}

A primeira linha da entrada contem um inteiro $N$, a quantidade de dias que durará a turnê. Cada uma das próximas $N$ linhas seguintes contém um inteiro $K[i]$, representando o valor do cachê de Pedro no dia $i$.

\section*{Saída}

A saída deve conter um único inteiro X, o valor máximo do cachê que Pedro pode receber na turnê.

\section*{Restrições}

\begin{itemize}
\item $1 \leq N \leq 10^5$
\item $1 \leq K[i] \leq 10^9$
\end{itemize}


\section*{Exemplos}

\exemplo
