Ernesto é um grande paraquedista que já acumula mais de 5000 saltos na sua carreira. Ele já se jogou de diferentes aviões, helicóptero, balão, já pousou no deserto, na praia, na amazônia e em uma variedade de outros locais que você possa imaginar. 

Ultimamente ele está com uma mania um tanto quanto diferente, antes de se jogar do avião ele decide que irá fazer apenas curvas de $G$ graus, seja para esquerda ou para direita, enquanto estiver navegando seu paraquedas.

Dado o valor $G$ e a sequência de curvas (E = esquerda, D = direita) feitas por Ernesto e assumindo que seu paraquedas não muda de direção (a não ser que ele faça propositalmente uma curva) e ele está sempre olhando fixamente para frente, é possível dizer que Ernesto teve uma visão completa ao seu redor (360 graus) durante sua navegação?


\section*{Entrada}

A primeira linha contém da entrada um inteiro $G$ e a segunda linha uma string de tamanho $S$ (composta apenas pelos caracteres 'D' ou 'E') representando a sequência de curvas feitas por Ernesto.

\section*{Saída}

Exiba o caracter $'S'$ caso Ernesto tenha tido uma visão completa ao seu redor durante sua navegação ou $'N'$ caso contrário.

\section*{Restrições}

\begin{itemize}
\item $1 \leq G \leq 360$
\item $1 \leq S \leq 10^5$
\end{itemize}

\section*{Exemplos}

\exemplo