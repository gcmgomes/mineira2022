% To add an image or include a .tex file you need to add
% \CWD
% to the relative (to the main document) path.
%
% Example:
% \begin{figure}
%   \centering
%   \includegraphics{\CWD/images/example.pdf}
% \end{figure}

Benedita é uma jovem muito criativa e interessada em ciência. Desde criança, ela aprendeu que o conhecimento racional
que a humanidade adquiriu ao longo da história permitiu com que a qualidade de vida melhorasse,
e que descobertas ainda maiores fossem feitas.

Um dos feitos mais inspiradores par Benedita foi a expedição à Lua de 1969, resultado da cooperação
de engenheiros e cientistas de várias áreas do conhecimento. Após muito trabalho, o foguete Apollo 11 levou astronautas
de uma base na Terra por uma viagem de vários dias, com partes muito críticas, como a decolagem, a aproximação e pousos lunares,
a operação do módulo lunar, e a reentrada na atmosfera terrestre.

Nessa viagem, os astronautas deixaram um arranjo de espelhos no solo lunar, capaz de refletir raios laser emitidos a partir da terra.
Quando soube disso, Benedita imediatamente começou a montar o equipamente apropriado em seu quintal e chamou todos os seus vizinhos
para acompanharem o experimento.
Para sua surpresa, ela descobriu que vários deles não acreditam na chegada da humanidade à Lua, e vão só causar transtornos
em sua festa laser.
Para evitar a fadiga, Benedita pediu que você escreva um programa que diz se uma pessoa deve ser cortada do evento.
Ela tem uma lista de vizinhos, composta apenas por zeros e uns.
Uma linha tem a entrada $0$ se o vizinho não acredita no pouso lunar; caso contrário a linha tem uma entrada $1$.
Dada uma única linha da lista, diga se o vizinho deve ou não ser cortado do conjunto de convidados!

\section*{Entrada}

A entrada possui uma linha contendo um valor inteiro $V$, a entrada na lista de Benedita.

\section*{Saída}

A saída deve possuir um único inteiro $A$, que deve ser $1$ se o vizinho deve ser bloquado e $0$ se o vizinho deve ser liberado.

\section*{Restrições}

\begin{itemize}
	\item $0 \leq V \leq 1$
\end{itemize}


\section*{Exemplos}

\exemplo
