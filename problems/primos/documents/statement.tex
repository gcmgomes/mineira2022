Um dos principais passatempos nas escolas da Bacônia moderna é o estudo da teoria dos números. O que poucos sabem é
que toda essa popularidade se deve aos esforços de Bacon -- o Grafo -- de promover o conhecimento e o avanço
científico durante seu reinado, principalmente na matemática e sua paixão pessoal, a combinatória.

A mais nova febre entre as jovens capivaras é o Jogo do Gato, que rapidamente tem ocupado espaço na Bacônia devido a sua
natureza competitiva.
Seu crescimento tem sido tão rápido que já há uma maratona planejada para o próximo ano, que promete desafiar até as capivaras
mais bem preparadas do reino. O jogo é disputado entre duas capivaras, e funciona em três fases.
Na primeira fase, uma capivara é designada como a numeróloga e a outra como o gato.
Na segunda fase, a numeróloga escolhe um número $N$ enquanto, ao mesmo tempo, o gato constroi uma lista $P$ de números primos.
No início da terceira e última fase, as capivaras revelam suas escolhas e começa a disputa.
Mas como ela funciona? O objetivo de cada capivara é encontrar uma forma de dividir $N$ em uma sequência de partes
de tal forma que cada parte é divisível por pelo menos um elemento de $P$. Uma parte é uma sequência contígua de dígitos de $N$.
Alguns minutos após o início da terceira fase, o juiz apita e as competidoras devem mostrar suas sequências, ou declarar que é impossível
formar uma sequência.
É declarada a vencedora do desafio a capivara que tiver a menor sequência em ordem lexicográfica; em caso de empate,
inicia-se uma nova rodada a partir da primeira fase.

Para tudo ficar mais claro, vamos a um exemplo com $N = 227$ e $P = \{2, 3, 7\}$. As três formas de partir $N$, ordenadas lexicograficamente,
são $\{\{2,2,7\}, \{2,27\}, \{22,7\}\}$. Dessa forma, se uma capivara escolheu $\{2, 2, 7\}$ e a outra escolheu $\{2, 27\}$, a primeira é declarada
a vencedora.

Os juizes da competição estão cansados de ter de resolver os casos esdrúxulos dos comptetidores manualmente a cada rodada, e pediram para você
escrever um programa que, dados $N$ e $P$, retorne a primeira sequência da ordem lexicográfica. No nosso exemplo, o seu programa deve retornar
$\{2, 2, 7\}$.

\section*{Entrada}

A linha da entrada contem dois inteiros, $N$ e $K$, onde $K$ é o comprimento do conjunto de primos $P$.
A segunda linha da entrada contém $K$ inteiros $p_1, \dots, p_K$, separados por espaços, que correspondem aos elementos de $P$.

\section*{Saída}

Caso seja impossível construir uma sequência, a única linha da saída deve conter o inteiro $-1$.
Caso contrário, a primeira linha da saída deve conter um único inteiro $M$: o tamanho da primeira sequência da ordem lexicográfica.
A segunda linha deve conter $M$ inteiros, separados por espaços, que correspondem às partes da sequência.
 

\section*{Restrições}

\begin{itemize}
    \item $1 \leq N \leq 10^{1000} - 1$
    \item $1 \leq K \leq 50$
    \item $2 \leq p_i \leq 10^{6}$
\end{itemize}


\section*{Exemplos}

\exemplo
