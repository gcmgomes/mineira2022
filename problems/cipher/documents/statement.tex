A cifra de transposição Rail Fence é um clássico algoritmo de criptografia. A ideia é cifrar uma mensagem escrevendo cada caracter diagonalmente para baixo e para cima em sucessivos trilhos de uma cerca imaginária. Um verdadeiro zig-zag! Depois, basta obter a sequência horizontal de caracteres, da esquerda para a direita, de cima para baixo, para formar a mensagem cifrada, sempre ignorando espaços e pontuação.
Por exemplo, para encriptar a mensagem: “QUEM TE CONHECE NAO ESQUECE JAMAIS” em 3 trilhos, escreva o texto como:

Q . . . T . . . N . . . E . . . E . . . E . . . A . . . S 
. U . M . E . O . H . C . N . O . S . U . C . J . M . I . 
. . E . . . C . . . E . . . A . . . Q . . . E . . . A . . 


Então, leia o texto horizontalmente, da esquerda para direita, de cima para baixo, ignorando espaços e pontuação. O resultado é o seguinte texto cifrado:

QTNEEEASUMEOHCNOSUCJMIECEAQEA

Missão:
Escreva uma solução para obter a mensagem original a partir da mensagem cifrada M e do número de trilhos N utilizados para codificá-la.

\section*{Entrada}

A primeira linha da entrada contém uma única string, $M$, a mensagem que deve ser decodificada. A segunda linha da entrada contém um único inteiro $T$, o número de trilhos que evem ser usados para decodificar a mensagem.

\section*{Saída}

A saída deve conter uma única string $P$, representando a mensagem decodificada.

\section*{Restrições}

\begin{itemize}
\item $1 \leq |M| \leq 2000$
\item $1 \leq |T| \leq |M|$
\end{itemize}

É garantido que $M$ contém apenas letras.

\section*{Exemplos}

\exemplo
