A cifra de transposição Rail Fence é um clássico algoritmo de criptografia. A ideia é cifrar uma mensagem escrevendo cada caractere diagonalmente
para baixo e para cima em sucessivos trilhos de uma cerca imaginária.
Um verdadeiro zig-zag!
Depois, basta obter a sequência horizontal de caracteres, da esquerda para a direita, de cima para baixo, para formar a mensagem cifrada, sempre
ignorando espaços e pontuação.
Por exemplo, para criptografar a mensagem: \texttt{“QUEM TE CONHECE NAO ESQUECE JAMAIS”} em 3 trilhos, escreva o texto como:

%\begin{center}
%\small
%\begin{verb} 
%Q . . . T . . . N . . . E . . . E . . . E . . . A . . . S
%\end{verb} 
%\begin{verb}
%. U . M . E . O . H . C . N . O . S . U . C . J . M . I . 
%\end{verb} 
%\begin{verb}
%. . E . . . C . . . E . . . A . . . Q . . . E . . . A . . 
%\end{verb}
%\end{center}



\begin{center}
\small
\begin{verb} 
Q  .  .  .  T  .  .  .  N  .  .  .  E  .  .  .  E  .  .  .  E  .  .  .  A  .  .  .  S
\end{verb} 
\begin{verb}
.  U  .  M  .  E  .  O  .  H  .  C  .  N  .  O  .  S  .  U  .  C  .  J  .  M  .  I  . 
\end{verb} 
\begin{verb}
.  .  E  .  .  .  C  .  .  .  E  .  .  .  A  .  .  .  Q  .  .  .  E  .  .  .  A  .  . 
\end{verb}
\end{center}


\normalsize
Então, leia o texto horizontalmente, da esquerda para direita, de cima para baixo, ignorando espaços e pontuação. O resultado é o seguinte texto cifrado:

\begin{verbatim}
QTNEEEASUMEOHCNOSUCJMIECEAQEA
\end{verbatim}

Sua missão é escrever uma solução para obter a mensagem original, ignorando espaços e pontuação, a partir da mensagem cifrada $C$ e do número de trilhos
$N$ utilizados para codificá-la.

\section*{Entrada}

A primeira linha da entrada contém uma única string, $C$, que representa a mensagem que deve ser decodificada. 
A segunda linha da entrada contém um único inteiro, $N$, que é o número de trilhos que devem ser usados para decodificar a mensagem.

\section*{Saída}

A saída deve conter uma única string $M$, representando a mensagem decodificada.

\section*{Restrições}

\begin{itemize}
\item $1 \leq |C| \leq 2000$
\item $1 \leq N \leq |C|$
\end{itemize}

É garantido que $C$ contém apenas letras maiúsculas.

\section*{Exemplos}

\exemplo
