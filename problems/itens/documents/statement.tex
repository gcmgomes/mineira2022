Alice e Bob resolveram dar um tempo da criptografia.
Enquanto estavam aproveitando as merecidas férias em uma das belas cachoeiras de Minas Gerais, eles
conheceram um ex-competidor da Maratona Mineira de Programação, que explicou em detalhes como a competição funcionava.
Infelizmente, Alice e Bob não são mais elegíveis para competir. Porém, eles queriam muito participar de alguma forma.

Bob pensou em propor problemas para a prova, mas Alice rapidamente percebeu que não seria uma boa ideia pois os competidores
saberiam que se trataria de um problema de criptografia assim que vissem os nomes dos autores. Então, sugeriram uma forma
de deixar a competição ainda mais interessante. Toda vez que algum time resolver um problema, o time vai receber uma notificação
no computador e, alguns segundos depois, o balão do problema amarrado a um peso será solto do alto em algum ponto do teto do salão
onde a prova estiver sendo realizada. Para poderem ficar com o balão, um dos membros do time tem que conseguir pegar o balão antes
que ele caia no chão. 

Os organizadores da prova acharam a ideia muito interessante, e estão pensando em implementá-la nas próximas edições.
Antes disso, precisam avaliar cuidadosamente todos os riscos e implicações. Por exemplo, algum competidor correndo desesperado
para pegar um balão poderia tropeçar em um cabo de rede. Então, talvez seja melhor usar uma rede sem fio. Outra questão importante
é o impacto na felicidade dos competidores. Os organizadores acreditam que felicidade dos competidores está ligada ao número de
balões que eles levam pra casa. Por isso, resolveram fazer uma análise para determinar a probabilidade de um competidor conseguir
pegar um balão.

O salão da prova foi modelado como um tabuleiro $N \times M$. Cada célula pode estar vazia ou conter algum obstáculo como, por exemplo, uma mesa.
O balão é solto diretamente acima de uma célula vazia escolhida aleatoriamente com probabilidade uniforme, e leva $T$ segundos para cair. 
O competidor se encontra na célula da linha $c_i$ e coluna $c_j$ e consegue se deslocar verticalmente ou horizontalmente para células vizinhas
vazias. O competidor leva $1$ segundo para se mover de uma célula para outra. 
Se o competidor chegar em uma célula ao mesmo tempo que o balão cair, ele consegue pegar o balão.
Como os competidores são muito espertos, se existir mais de um caminho que permite que o competidor pegue o balão, ele vai escolher o menor deles.
A organização quer saber qual é probabilidade de que o competidor consiga pegar o balão antes que ele caia. Como a
organização estava muito ocupada crimpando cabos de rede, enchendo balões e comprando salgados, resolveram pedir para que você calcule essa probabilidade, o que deixou Bob muito feliz pois agora ele ajudou a colocar um problema na prova.


\section*{Entrada}

A primeira linha da entrada contém cinco inteiros separados por espaço $N$, $M$, $T$, $c_i$ e $c_j$, onde $N$ e $M$ representam respectivamente o
número de linhas e colunas do grid, $T$ representa o tempo em segundos que o balão leva para cair, e $c_i$ e $c_j$ representam respectivamente a linha e a coluna do grid da célula onde o competidor se
encontra.

\section*{Saída}
A saída deve conter um única linha com dois inteiros $A$ e $B$ separados por espaço, de forma que $\dfrac{A}{B}$ represente a probabilidade desejada, $B > 0$ e o maior fator comum entre $A$ e $B$ seja $1$.
\section*{Restrições}

\begin{itemize}
    \item $1 \leq N \leq 1000$.
    \item $1 \leq M \leq 1000$.
    \item $1 \leq T \leq {10}^6$.
    \item $1 \leq c_i \leq N$.
    \item $1 \leq c_j \leq M$.
\end{itemize}


\section*{Exemplos}

\exemplo
