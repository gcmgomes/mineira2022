Madalena é uma cachorrinha que adora fazer bagunça e andar em círculos. Sempre que seu dono, Dâmiko, chega em casa, encontra vários brinquedos espalhados pela casa e a cachorrinha Madalena girando e girando pelos cômodos.
Dâmiko adora encontrar padrões e descobriu que os brinquedos não ficam espalhados de forma aleatória:

\begin{itemize}
  \item Um brinquedo fica sempre em cima de um azulejo e Madalena brinca apenas com os brinquedos do azulejo que ela está no momento.
  \item Quando Madalena começa a brincar, ela pega o máximo possível de brinquedos do azulejo em que está e distribui de forma igualitária para os azulejos adjacentes. Por exemplo, se existem 20 brinquedos no azulejo atual e 8 azulejos adjacentes, 2 brinquedos irão para cada um deles e 4 continuarão no azulejo atual.
  \item Após distribuir os brinquedos, Madalena se move para um azulejo adjacente. Ela escolhe aquele que tem mais brinquedos (porque assim é muito mais divertido) e recomeça o processo de distribuí-los. Se existe mais de um possível próximo azulejo com o mesmo número de brinquedos, ela escolhe o primeiro do sentido horário que encontrar (iniciando pelo azulejo à sua esquerda).
\end{itemize}

A brincadeira acaba quando Madalena brinca em $L + C + 1$ azulejos.

Dado o azulejo inicial onde Madalena se encontra, Dâmiko deseja saber a maior quantidade de brinquedos em um único azulejo ao final da brincadeira.

\section*{Entrada}

A primeira linha da entrada contém dois valores inteiros $L$ e $C$ que representam as quantidade de linhas e colunas de azulejos. Em seguida, $L$ linhas conterão os elementos de uma matriz $M$ de dimensão $L \times C$ com elementos inteiros representando a quantidade de brinquedos em cada um dos azulejos.
Por fim, será dada uma última linha contendo dois valores inteiros $A$ e $B$ que representam a linha e a coluna, respectivamente, onde Madalena inicia sua brincadeira.

\section*{Saída}

A saída deve conter o maior número de brinquedos em um único azulejo ao final da brincadeira.


\section*{Restrições}

\begin{itemize}
\item $1 \leq L, C \leq 500$.
\item $0 \leq M_{ij} \leq 10^6$.
\item $1 \leq A \leq L$.
\item $1 \leq B \leq C$.
\end{itemize}


\section*{Exemplos}

\exemplo

