Cirila adora jogos de tabuleiro, tanto pelas noites de diversão que eles proporcionam para seu grupo de amigos quanto pelos quebra cabeças que ela tem de resolver para jogar de forma mais eficiente possível e vencer.
Sua paixão por eles é tão grande que ela resolveu criar o próprio jogo: Bloomhaven. Rapidamente, ela descobriu que garantir o equilíbrio entre os diferentes componentes é muito difícil, e que resolver vários casos de teste na mão é cansativo e demanda muito tempo.
A última mecânica de jogo que ela precisa implementar é o sistema de feitiços, e Cirila decidiu fazer ele bem diferente de todos os outros que já viu.
Ele funciona da seguinte forma: cada personagem possui um disco com $N$ símbolos, e entre dois símbolos consecutivos está um encantamento com um determinado poder.
Para realizar uma magia, é preciso escolher um subconjunto não vazio de encantamentos. Cada encantamento escolhido ativa os dois símbolos adjacentes a ele. Para que a magia seja considerada válida, é preciso que: (i) cada símbolo seja ativado por no máximo um dos encantamos escolhidos, (ii) os símbolos ativos formem uma sequência contígua no disco.
O poder de uma magia é soma dos poderes dos encantamentos escolhidos para realizá-la.
Cirila gostaria de saber qual a magia mais poderosa que pode ser usada pelo personagem, mas não quer gastar mais tempo que o necessário fazendo essas contas.
Escreva um programa que ajude Cirila a resolver esse problema e, quem sabe, terá seu nome no livro de regras com um agradecimento especial!

\section*{Entrada}

A primeira linha da entrada contém um único inteiro $N$ o número de símbolos no disco do personagem.
A segunda linha contém $N$ inteiros, o $i$-ésimo deles, $X_i$, representa o poder do encantamento entre os símbolos $i$ e $(i \mod N) + 1$.

\section*{Saída}

A saída contem um único inteiro $M$, o poder máximo do feitiço que pode ser utilizado pelo personagem.

\section*{Restrições}

\begin{itemize}
\item $3 \leq N \leq 10^6$
\item $-10^5 \leq X_i \leq 10^5$
\end{itemize}


\section*{Exemplos}

\exemplo
