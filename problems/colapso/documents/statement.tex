Bibika trabalha no centro nacional de meteorologia onde sua função é estudar dados passados referentes a períodos chuvosos em diversas cidades a
fim de entender como cidades em estados de emergência afetam para o crescimento do país.

Para fins didáticos, iremos considerar que um país é uma grade retangular de tamanho $N$x$M$ onde cada posição desse retângulo representa uma cidade.
Após anos de estudos, Bibika está quase certa de que um país entra em colapso, por conta dos problemas em longos períodos de chuva, sempre que todas
as cidades contidas em um quadrado de lado $K$ declaram estado de emergência.

Querendo confirmar sua teoria, Bibika conta com sua ajuda para analisar mais alguns períodos chuvosos da história e descobrir qual foi o primeiro momento
em que o país entrou em colapso, dada a definição de colapso descrita anteriormente.

É válido destacar que uma cidade nunca entra em estado de emergência mais de uma vez e, uma vez em estado de emergência, ela permanece nesse estado para sempre.

\section*{Entrada}

A primeira linha da entrada contém quatro inteiros $N$, $M$, $K$, $Q$, representando, respectivamente, a quantidade de linhas e colunas do país,
o tamanho do lado do quadrado e a quantidade de dados de cidades que declararam estado emergência.
Cada uma das próximas $Q$ linhas contém três inteiros $A$, $B$, $D$ significando que no momento $D$ no tempo a cidade da posição $(A,B)$ declarou estado
de emergência.

\section*{Saída}

A saída deve conter um único inteiro: o primeiro instante em que seria possível dizer que o país entrou em colapso. 
Caso ele não tenha entrado em colapso em nenhum momento, exiba $-1$.

\section*{Restrições}

\begin{itemize}
\item $1 \leq N, M \leq 400$
\item $1 \leq K \leq min(N, M)$
\item $1 \leq Q \leq N \times M$
\item $1 \leq A \leq N$
\item $1 \leq B \leq M$
\item $1 \leq D \leq 10^9$
\end{itemize}


\section*{Exemplos}

\exemplo
