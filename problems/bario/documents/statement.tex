Tudo começou com uma grande explosão cósmica. Surgiram os quarks, os planetas, as nebulosas, as galáxias, e a Maratona Mineira de Programação.
Para a edição de 2022, os organizadores pensaram bastante em como fazer uma super edição do evento para comemorar o seu retorno. Uma das ideias mais
ousadas era o desenvolvimento de um jogo de plataforma de um dos personagens mais icônicos do mundo: o Bário.
Nesse novo lançamento, Bário se moveria em fases compostas por $N$ blocos.
Cada bloco pode ser sólido um buraco, e o objetivo do jogo é chegar do início da fase (o bloco $1$) até o final (o bloco $N$) sem cair em nenhum buraco.
Para evitar buracos, Bário pode pular.
Além disso, ele tem o poder de carregar suas botas de energia para pular buracos muito grandes.
Cada bloco sobre o qual Bário corre adiciona uma unidade de carga a seu equipamento.
Um pulo de carga $v$ faz com que Bário vá do bloco $b$ para o primeiro bloco sólido entre as posições $b+2$ e $b+v+1$;
ao aterrisar, Bário perde toda a energia de suas botinhas, retornando ao valor inicial de $1$.
O bloco onde Bário aterrisa e o primero bloco da fase não adicionam carga a suas botas.
Como vocês devem ter reparado, não há um novo jogo para vocês se divertirem, só esse problema, que é quase tão legal.
O comitê de prova se rebelou e não quis implementar as fases, pois achou muito difícil determinar se uma fase era completável.
Resolvam esse problema e talvez, quem sabe, podemos ter uma edição especial de Bário para Mineira de 2023.



\section*{Entrada}

A primeira linha da entrada contem um único inteiro $N$, que representa o número de blocos na fase.
A segunda linha contém $N$ caracteres. Caracteres $x$ representam blocos sólidos, enquanto caracteres $.$ representam buracos na fase. É garantido que o último bloco da fase é sólido.

\section*{Saída}

A saída contem um único inteiro $P$, o menor número de pulos que Bário pricasa dar para completar a fase, ou $-1$ caso seja impossível completar a fase.

\section*{Restrições}

\begin{itemize}
\item $2 \leq N \leq 5 * 10^5$
\end{itemize}


\section*{Exemplos}

\exemplo
