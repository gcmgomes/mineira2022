

Este caderno de tarefas é composto por \pageref{lastpage} páginas (não
contando a folha de rosto), numeradas 
de 1 a \pageref{lastpage}. Verifique se o caderno está completo.

\subsection*{Nome do programa}
Cada problema tem os possíveis nomes de arquivo fonte indicados abaixo do
título.  Soluções na linguagem C devem ser arquivos com sufixo \emph{.c};
soluções na linguagem C++ devem ser arquivos com sufixo \emph{.cc} ou
\emph{.cpp}; soluções na linguagem Java devem ser arquivos com sufixo
\emph{.java} e a classe principal deve ter o mesmo nome do arquivo fonte; e
soluções na linguagem Python devem ser arquivos com sufixo \emph{.py}.

\subsection*{Entrada}

\begin{itemize}
\item A entrada deve ser lida da entrada padrão.

\item A entrada consiste em exatamente um caso de teste, que é descrito usando uma
quantidade de linhas que depende do problema. O formato da entrada é como descrito
em cada problema. A entrada não contém nenhum conteúdo extra.

\item Todas as linhas da entrada, incluindo a última, terminam com o caractere de fim
de linha (\textbackslash n).

\item A entrada não contém linhas vazias.

\item Quando a entrada contém múltiplos valores separados por espaços, 
existe exatamente um espaço em branco entre dois valores consecutivos na mesma linha.

\end{itemize}

\subsection*{Saída}

\begin{itemize}

\item A saída deve ser escrita na saída padrão.

\item A saída deve respeitar o formato especificado no enunciado. A saída não deve
conter nenhum dado extra.

\item Todas as linhas da saída, incluindo a última, devem terminar com o caractere de fim
de linha (\textbackslash n).

\item Quando uma linha da saída apresentar múltiplos valores separados por espaços, deve haver
exatamente um espaço em branco entre dois valores consecutivos.

\item Quando a um valor da saída for um número real, use pelo menos o número de casas decimais correspondente
à precisão requisitada no enunciado.

\end{itemize}
