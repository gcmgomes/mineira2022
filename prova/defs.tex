%\documentclass[a4paper,11pt]{article}
\usepackage{amsmath}
\usepackage[brazil]{babel}
\usepackage[utf8]{inputenc}
\usepackage[T1]{fontenc}
\usepackage{indentfirst}
\usepackage{float}
\usepackage{fancyvrb}
\usepackage{pdftexcmds}
\usepackage{multicol}
\usepackage{amsmath}
\usepackage{hyperref}
\usepackage{ifthen}
\usepackage{listings}
\usepackage{epsfig}
\usepackage{tikz}

\setlength{\marginparwidth}{0pt}
\setlength{\oddsidemargin}{-0.25cm}
\setlength{\evensidemargin}{-0.25cm}
\setlength{\marginparsep}{0pt}

\setlength{\parindent}{0cm}
\setlength{\parskip}{5pt}

\setlength{\textwidth}{16.5cm}
\setlength{\textheight}{25.5cm}

\setlength{\voffset}{-1in}

\newcommand{\insereArquivo}[1]{
\ifnum0\pdffilesize{#1}>0
	\VerbatimInput[xleftmargin=0mm,numbers=none,obeytabs=true]{#1}\vspace{.5em}
\fi
}

%\newcommand{\textoDiversasInstancias}{A entrada é composta por
%diversos casos de teste. }
\newcommand{\textoDiversasInstancias}{}

%\newcommand{\arquivoProblema}[1]{\vspace{-0.3cm} \noindent {\em
%Arquivo: \texttt{#1.[c|cpp|java]} \\}}

\newcommand{\textoSaidaPadrao}{\vspace{0.2cm} \noindent \emph{A
saída deve ser escrita na saída padrăo. }}

\newcommand{\textoEntradaPadrao}{\vspace{0.2cm} \noindent \emph{A
entrada deve ser lida da entrada padrăo. }}

\newcommand{\incat}[1]{sample-#1.in}
\newcommand{\solcat}[1]{sample-#1.sol}
\newcounter{count}
\setcounter{count}{0}

\newcommand{\exemplo}{
\vspace{0.3cm}
{\small

\setcounter{count}{0}
% first check if sample.in is present (maratona)
\IfFileExists{\CWD/sample.in} {
  \begin{minipage}[c]{0.9\textwidth}
  \begin{center}
  \begin{tabular}{|l|l|} \hline
  \begin{minipage}[t]{0.5\textwidth}
  \bf{Exemplo de Entrada}
  \insereArquivo{\CWD/sample.in}
  \end{minipage}
  &
  \begin{minipage}[t]{0.5\textwidth}
  \bf{Saída para o Exemplo de Entrada}
  \insereArquivo{\CWD/sample.sol}
  \end{minipage}\\
  \hline
  \end{tabular}
  \end{center}
  \end{minipage} % leave next line empty

  }{} % IfFileExists

% then check if sample-n.in is present (obi)
\whiledo{\value{count}<10}{
  \addtocounter{count}{1}
  \IfFileExists{\CWD/\incat{\thecount}} {
  \begin{minipage}[c]{0.9\textwidth}
  \begin{center}
  \begin{tabular}{|l|l|} \hline
  \begin{minipage}[t]{0.5\textwidth}
  \bf{Entrada}
  \insereArquivo{\CWD/\incat{\thecount}}
  \end{minipage}
  &
  \begin{minipage}[t]{0.5\textwidth}
  \bf{Saída}
  \insereArquivo{\CWD/\solcat{\thecount}}
  \end{minipage}\\
  \hline
  \end{tabular}
  \end{center}
  \end{minipage} % leave next line empty

  }{} % IfFileExists
} % while
}
} % exemplo

\newcommand{\exemplolongoesquerda}{
\vspace{0.3cm}
{\small

\setcounter{count}{0}
% first check if sample.in is present (maratona)
\IfFileExists{\CWD/sample.in} {
  \begin{minipage}[c]{0.9\textwidth}
  \begin{center}
  \begin{tabular}{|l|l|} \hline
  \begin{minipage}[t]{0.5\textwidth}
  \bf{Exemplo de Entrada}
  \insereArquivo{sample.in}
  \end{minipage}
  &
  \begin{minipage}[t]{0.5\textwidth}
  \bf{Saída para o Exemplo de Entrada}
  \insereArquivo{sample.sol}
  \end{minipage}\\
  \hline
  \end{tabular}
  \end{center}
  \end{minipage} % leave next line empty

  }{} % IfFileExists

% then check if sample-n.in is present (obi)
\whiledo{\value{count}<10}{
  \addtocounter{count}{1}
  \IfFileExists{\CWD/\incat{\thecount}} {
  \begin{minipage}[c]{0.95\textwidth}
  \begin{center}
  \begin{tabular}{|l|l|} \hline
  \begin{minipage}[t]{0.88\textwidth}
  \bf{Entrada}
  \insereArquivo{\CWD/\incat{\thecount}}
  \end{minipage}
  &
  \begin{minipage}[t]{0.12\textwidth}
  \bf{Saída}
  \insereArquivo{\CWD/\solcat{\thecount}}
  \end{minipage}\\
  \hline
  \end{tabular}
  \end{center}
  \end{minipage} % leave next line empty

  }{} % IfFileExists
} % while
}
} % exemplolongoesquerda


%\newcommand{\incluir}[1]{
%\input{#1}
%}
